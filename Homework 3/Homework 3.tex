\documentclass[12pt]{article}
\usepackage[margin=1in]{geometry}
\usepackage{amsmath,amssymb,amsthm}
\renewcommand{\qedsymbol}{$\blacksquare$}

\newtheorem{theorem1}{Theorem}
\newtheorem{corollary}{Corollary}[theorem1]

\begin{document}
	
	\title{\Large Homework \#3 MATH270}
	\author{Luke Tollefson}
	\date{Wednesday, February 8th}
	
	\maketitle
	\begin{enumerate}
		\item Prove that if $n$ is an integer, 
		then $4n^2+4n+8$ is an even integer. 
		What method of proof did you use?
		\begin{proof}
		We will prove this directly. By definition, if an integer $n$ is even, then $n=2m,m\in {\mathbb{Z}}$. Since $4n^2+4n+8=2(2n^2+2n+4)$, then by definition $4n^2+4n+8$ is an even integer.
		\end{proof}
	
		\item Complete the proof for the theorem:
		\begin{theorem1}
			Let $x$ and $y$ be real numbers. If $xy>1/2$ then $x^2+y^2>1$.
		\end{theorem1}
		\begin{proof}
			The proof will proceed by considering the contrapositive. So suppose $x^2+y^2\leq 1$. Now we know that $(x^2-y^2)\geq 0$ (this is always true).
			\begin{align*}
			1&\leq (x-y)^2+1\\
			x^2+y^2&\leq (x-y)^2+1\\
			x^2+y^2&\leq x^2-2xy+y^2+1\\
			0&\leq -2xy+1\\
			2xy&\leq 1\\
			xy&\leq \frac{1}{2}
			\end{align*}
			This shows the contrapositive is true. So then the theorem is true.
		\end{proof}
		\newpage
		\item Prove that $\sqrt{3}$ is not rational. Before we prove that, we will prove that if $n^2$ is divisible by $3$, then $n$ is divisible by $3$.
		\begin{proof}
			We will prove that if $3\mid n^2$, then $3\mid n$ by proving the contrapositive. If $3\nmid n$, then by definition there is not integer $m$ such that $n=3m$. Then there should also be no $m$ such that $n^2=3(3m^2)$, which means $3\nmid n^2$.
		\end{proof}
		\begin{proof}
			We will prove this by contradiction. Assume $\sqrt{3}$ is rational. Then $\frac{p}{q}=\sqrt{3}$, $p,q\in \mathbb{Z}$ where $p$ and $q$ are coprime. Squaring both sides yields $\frac{p^2}{q^2}=3$. So $p^2=3q^2$. By the previous proof, since $3\mid p^2$, then $3\mid p$. So $p=3m$, then $p^2=9m^2=3q^2$, or $q^2=3m^2$. Since $3\mid q^2$, then $3\mid q$. Now we have shown that $3$ divides both $p$ and $q$, showing they have a common factor of $3$, contradicting the assumption that they were coprime.\\
			This contradiction proves that $\sqrt{3}$ must not be rational.
		\end{proof}
	
		\item Let $x$ be a real number.\\
			(a) Prove $-|x|\leq x\leq |x|$*
			\begin{proof}
				We will prove this by looking at each possible case. The definition of the absolute value function
				\[ |x| = \left\{
				\begin{array}{rl}
				x, & \  x \geq 0 \\
				-x, & \  x < 0
				\end{array}\right.
				\]
				there are two cases, $x\geq 0$ and $x<0$.\\
				Case: If $x\geq 0$, then $|x|=x$. Substituting for $|x|$ into (*) yields $-x\leq x\leq x$, which is true.\\
				Case: If $x<0$, then $|x|=-x$. Substituting for $|x|$ into (*) yields $x\leq x\leq -x$. This is true. The largest term is positive, where as the two left ones are negative.
			\end{proof}
		
			(b) Let $a\geq 0$. Prove $|x|\leq a$ if and only if $-a\leq x\leq a$.
			\begin{proof}
				First we will prove that if $|x|\leq a$, then $-a\leq x\leq a$ by cases.\\
				Case: If $x\geq 0$, then $x\leq a$, which means $-a\leq 0\leq x\leq a$. And that is true.\\
				Case: If $x<0$, then $-x\leq a\implies x\geq -a$, then $-a\leq x< 0<a$\\
				Next we will prove that if $-a\leq x\leq a$, then $|x|\leq a$, again by cases.\\
				Case: $-a<0\leq x\leq a$. In this case $x$ is positive, so $|x|=x$, therefore $|x|\leq a$.\\
				Case: $-a\leq x<0<a\implies -a\leq x\implies -x\leq a$. Since $x<0$, then $|x|=-x$, therefore $|x|\leq a$.
			\end{proof}
		\newpage
			(c) Prove the following.
			\begin{theorem1}
				Let $x$ and $y$ be real numbers. Then
				$|x+y|\leq|x|+|y|$
			\end{theorem1}
			\begin{proof}
				\begin{align*}
					-|x|&\leq x\leq |x|\\
					-|y|&\leq y\leq |y|
				\end{align*}
				Summing these two formulas yields
				\[
					-(|x|+|y|)\leq x+y\leq |x|+|y|
				\]
				Implying
				\[
					|x+y|\leq ||x|+|y||
				\]
				$|x|+|y|$ is always positive, so $||x|+|y||=|x|+|y|$, therefore proving
				\[
					|x+y|\leq |x|+|y|
				\]
			\end{proof}
			
			(d) Prove the following corollary to Theorem 2.
			\begin{corollary}
				For any $x,y\in \mathbb{R}$, $||x|-|y||\leq |x-y|$.
			\end{corollary}
			\begin{proof}
				\begin{align}
					|x|&=|x-y+y|\nonumber \\
					|x-y+y|&\leq |x-y+y|\nonumber \\
					|x-y+y|&\leq |x-y|+|y|\nonumber \\
					|x-y+y|-|y|&\leq |x-y|\nonumber \\
					|x|-|y|&\leq |x-y|
				\end{align}
				Proving a second result
				\begin{align}
					x&\leq |x|\nonumber\\
					y&\leq |y|\nonumber\\
					x-y&\leq |x|-|y|\nonumber\\
					-|x-y|&\leq x-y\qquad \text{by 4a}\nonumber\\
					-|x-y|&\leq |x|-|y|
				\end{align}
				 Combining (1) and (2) gives
				 \[
				 	-|x-y|\leq |x|-|y|\leq |x-y|
				 \]
				 Using the result from 4b
				 \[
				 	||x|-|y||\leq |x-y|
				 \]
			\end{proof}
	

	\end{enumerate}

	
\end{document}