\documentclass[12pt]{article}
\usepackage[margin=1in]{geometry}
\usepackage{amsmath,amssymb,amsthm}
\usepackage{enumitem}
\setlist{parsep=0pt,listparindent=\parindent}
\usepackage[mathscr]{euscript}
\let\euscr\mathscr \let\mathscr\relax% just so we can load this and rsfs
\usepackage[scr]{rsfso}
\newcommand{\powerset}{\raisebox{.15\baselineskip}{\Large\ensuremath{\wp}}}

\newtheorem{theorem}{Theorem}

\begin{document}
	
	\title{\Large Homework \#5 MATH270}
	\author{Luke Tollefson}
	\date{Wednesday, February 27th}
	
	\maketitle
	\begin{enumerate}
		\item The theorem is correct and properly proved directly. Here is an alternate proof.
		
		\begin{theorem}
			For any sets $A$, $B$, and $C$, if $A\setminus B\subseteq C$ and $A\not\subseteq C$ then $A\cap B\not=\emptyset$.
		\end{theorem}
		\begin{proof}
			Consider the contrapositive \textit{if $A\cap B=\emptyset$ then $A\setminus B\not\subseteq C$ or $A\subseteq C$}. The consequence states $A\not\subseteq C$ or $A\subseteq C$, which is always true. Hence, the implication is true.
		\end{proof}
		
		\item If $A_x=[-x,x]$, what is $\bigcap_{x\in \mathbb{R^+}}A_x$ and $\bigcup_{x\in \mathbb{R^+}}A_x$?
		\begin{align*}
			\bigcap_{x\in \mathbb{R^+}}A_x=\{0\}
		\end{align*}
		\begin{proof}
			If $y\in \bigcap_{x\in \mathbb{R^+}}A_x$, then $y\in A_x$ for all $x\in \mathbb{R^+}$. Considering the interval $A_x=[-x,x]$, if we have some number $\epsilon\in \mathbb{R^+}$, we can always choose a smaller number $\frac{\epsilon}{2}\in \mathbb{R^+}$. Therefore, the only number to be in all intervals $[-x,x]$ is 0. And so $y\in \{0\}$, thus $\bigcap_{x\in \mathbb{R^+}}A_x\subseteq\{0\}$.
			
			
			
			Now let $y\in \{0\}$. For each $x\in\mathbb{R^+}$, we may write $-x\leq y\leq x$. And so $y\in A_x$ for all $x\in\mathbb{R^+}$. This shows $y\in \bigcap_{x\in \mathbb{R^+}}A_x$, hence $\{0\}\subseteq \bigcap_{x\in \mathbb{R^+}}A_x$.
			
			Combining the two implications yields the equality, $\bigcap_{x\in \mathbb{R^+}}A_x=\{0\}$.
		\end{proof}
		\begin{align*}
			\bigcup_{x\in \mathbb{R^+}}A_x=(-\infty,\infty)
		\end{align*}
		\begin{proof}
			If $y\in \bigcup_{x\in \mathbb{R^+}}A_x$, then $y\in [-x,x]$ for some $x\in \mathbb{R^+}$. Since $x$ can be arbitrarily large, $y\in (-\infty,\infty)$, therefore $\bigcup_{x\in \mathbb{R^+}}A_x\subseteq(-\infty,\infty)$.
			
			Now if $y\in (-\infty,\infty)$, then for some $x\in\mathbb{R^+}$ we may have $y\in [-x,x]$. So $y\in A_x$, hence $(-\infty,\infty)\subseteq \bigcup_{x\in \mathbb{R^+}}A_x$.
			
			These two arguments combined to form the equality $\bigcup_{x\in \mathbb{R^+}}A_x=(-\infty,\infty)$.
		\end{proof}
	
		\newpage
	
		\item Prove the following set inclusion 
		\begin{align*}
			\bigcup_{b\in \mathbb{R^+}} \{(x,y)\in \mathbb{R}^2:x+y=b\} \subseteq \bigcap_{s\in \mathbb{R^-}} \{(x,y)\in \mathbb{R}^2:x+y>s \}.
		\end{align*}
		\begin{proof}
			Let $z\in \bigcup_{b\in \mathbb{R^+}} \{(x,y)\in \mathbb{R}^2:x+y=b\}$, then $z\in \{(x,y)\in \mathbb{R}^2:x+y=b\}$ for some $b\in \mathbb{R^+}$. Now for all $s\in \mathbb{R^-}$ we have $z\in \{(x,y)\in \mathbb{R}^2:x+y=b>s\}$, or $z\in \{(x,y)\in \mathbb{R}^2:x+y>s\}$. Therefore $z\in\bigcap_{s\in \mathbb{R^-}} \{(x,y)\in \mathbb{R}^2:x+y>s \}$ and the theorem is complete.
		\end{proof}
	
		\item (a) False, let $A=\emptyset$, $\{\emptyset\}\not\subseteq \emptyset$\\
		(b) True, $\emptyset$ is a subset of any set. However, $\emptyset$ is never equal to $\mathscr{P}(A)$ because $\mathscr{P}(A)$ is a set that always contains at least one element (the element being $\emptyset$).\\
		(c) False, $\{\emptyset \}\in\mathscr{P}(A)$, but $\{\emptyset \}\not\subseteq \{x,y \}$.\\
		(d) False, $\mathscr{P}(\{\{x,y \} \})=\{\emptyset, \{\{x,y \} \}\}$ \\
		(e) False, let $A=\emptyset$, $\mathscr{P}(\emptyset)=\{\emptyset \}, \{\emptyset \}\not\in\{\emptyset \}$
		
		\item Show that
		\begin{align}
			\mathscr{P}(A)\cup\mathscr{P}(B)\subseteq\mathscr{P}(A\cup B),
		\end{align}
		but
		\begin{align}
		\mathscr{P}(A)\cup\mathscr{P}(B)\not=\mathscr{P}(A\cup B).
		\end{align}
		\begin{proof}
			Let $x\in \mathscr{P}(A) \cup\mathscr{P}(B)$, then $x\in\mathscr{P}(A)$ or $x\in\mathscr{P}(B)$. Assume that $x\in\mathscr{P}(A)$, then $x\subseteq A$. It follows that $x\subseteq A\cup B$, hence $x\in\mathscr{P}(A\cup B)$. Similarly, assume $x\in\mathscr{P}(B)$, then $x\subseteq B$. Now $x\subseteq A\cup B$, therefore $x\in \mathscr{P}(A\cup B)$. This proves (1).
			
			Now we can show (2) by counter example. If $A=\{1\}$ and $B=\{2\}$, then $\mathscr{P}(A)\cup\mathscr{P}(B)=\{\emptyset, \{1\}, \{2\} \}$ and $\mathscr{P}(A\cup B)=\{\emptyset, \{1\}, \{2\}, \{1,2\} \}$. This shows the two sets are not equivalent.
		\end{proof}
	
		\item Let $\{A_\alpha:\alpha\in I \}$ be a nonempty indexed collection of sets. Prove that
		\begin{align*}
			\mathscr{P}\left(\bigcap_{\alpha\in I}A_\alpha\right)=\bigcap_{\alpha\in I}\mathscr{P}(A_\alpha)
		\end{align*}
		\begin{proof}
			Let $x\in \mathscr{P}(\bigcap_{\alpha\in I}A_\alpha)$, then $x\subseteq \bigcap_{\alpha\in I}A_\alpha$. Let $z\in x$, then $z\in \bigcap_{\alpha\in I}A_\alpha$. By definition, $z\in A_\alpha$ for all $\alpha\in I$. Since $z\in x$ implied $z\in A_\alpha$, then $x\subseteq A_\alpha$. Therefore, $x\in \mathscr{P}(A_\alpha)$ for all $\alpha\in I$. Hence $x\in \bigcap_{\alpha\in I}\mathscr{P}(A_\alpha)$.
			
			Now let $x\in \bigcap_{\alpha\in I}\mathscr{P}(A_\alpha)$. Therefore, $x\in \mathscr{P}(A_\alpha)$ for all $\alpha\in I$. Now, by definition, $x\subseteq A_\alpha$. If we let $z\in x$, then $z\in A_\alpha$, and therefore $z\in \bigcap_{\alpha\in I}A_\alpha$. Since $z\in x$ implied $z\in \bigcap_{\alpha\in I}A_\alpha$, then $x\subseteq \bigcap_{\alpha\in I}A_\alpha$. It follows $x\in \mathscr{P}(\bigcap_{\alpha\in I}A_\alpha)$.
			
			Since containment was proved in both directions, the two sets are equivalent.
		\end{proof}
	\end{enumerate}
	

\end{document}