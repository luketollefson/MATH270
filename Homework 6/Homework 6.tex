\documentclass[12pt]{article}
\usepackage[margin=1in]{geometry}
\usepackage{amsmath,amssymb,amsthm}
\usepackage{enumitem}
\setlist{parsep=0pt,listparindent=\parindent}

\begin{document}
	
	\title{\Large Homework \#6 MATH270}
	\author{Luke Tollefson}
	\date{Wednesday, March 6th}
	
	\maketitle
	
	
	\begin{enumerate}
		\item (a) Prove $A\times (B\cup C)=(A\times B)\cup (A\times C)$.
		\begin{proof}
			Let $x\in A\times (B\cup C)$, then $x=(y,z)$, where $y\in A$ and $z\in B\cup C$. In other words, $y\in A$ and $z\in B$, or the case $y\in A$ and $z\in C$. Thus $x\in A\times B$ or $x\in A\times C$. Therefore $x\in (A\times B)\cup (A\times C)$.
			
			To prove containment in the other direction, let $x\in (A\times B)\cup (A\times C)$. Therefore $x\in A\times B$ or $x\in B\times C$. Then $x=(y,z)$, where $y\in A$ and $z\in B$ or the case $y\in A$ and $z\in B$. Hence $y\in A$ and $z\in B\cup C$. Therefore $x\in A\times (B\cup C)$.
			
			Since containment was shown in both directions, equality is proved.
		\end{proof}
	
		\noindent (b) Prove $A\times (B\cap C)=(A\times B)\cap (A\times C)$
		
		\begin{proof}
			Let $x\in A\times(B\cap C)$. Then $x=(y,z)$ where $y\in A$ and $z\in B\cap C$. Hence $z\in B$ and $z\in C$. So it's the case $y\in A$ and $z\in B$, and the case $y\in A$ and $z\in C$. Therefore $x\in (A\times B)\cap (A\times C)$.
			
			Now let $x\in (A\times B)\cap (A\times C)$. So $x=(y,z)$ where the case $y\in A$ and $z\in B$ and the case $y\in A$ and $z\in C$. Hence, $z\in B\cap C$. Since $y\in A$ and $z\in B\cap C$, $x\in A\times (B\cap C)$.
			
			Containment was proved in both directions showing equality.
		\end{proof}
	
		\item On the paper stapled.
		
		\item Let $X=\{1,2,3,4,5\}$. (a) define an equivalence relation. (b) define a reflexive, but not symmetric nor transitive relation. (c) define a symmetric, but not reflexive nor transitive relation. (d) define a transative, but not reflexive nor symmetric relation.
		
		\noindent (a) $x\sim y\iff T$
		
		\noindent (b) $x\sim y\iff x-y$ is odd and $x-y\geq 0$
		
		\noindent (c) $x\sim y\iff x\neq y$
		
		\noindent (d) $x\sim y\iff x\mid y$ and $x\neq y$
		
		\newpage
		
		\item Define a relation $\mathbb{R}$ as follows: $x\sim y$ if and only if $x^2-y^2\in \mathbb{Z}$. Prove its and equivalence relation and give five different real numbers in the equivalence class $E_{\sqrt{2}}$.
		
		\begin{proof}
			To prove that it is an equivalence relation we will show it is reflexive, symmetric, and transitive.
			
			Let $x\in \mathbb{R}$. Then $x^2-x^2=0\in \mathbb{Z}$. Hence $x\sim x$.
			
			If $x\sim y$, then $x^2-y^2\in \mathbb{Z}$. Negating yields another integer $y^2-x^2$. Since $y^2-x^2\in \mathbb{Z}$, $y\sim x$.
			
			If $x\sim y$ and $y\sim z$, then $x^2-y^2\in\mathbb{Z}$ and $y^2-z^2\in\mathbb{Z}$. Adding $x^2-y^2$ and $y^2-z^2$ yields another integer $x^2-z^2$. So $x^2-z^2\in\mathbb{Z}$, therefore $x\sim z$.
			
			Hence, the relation is indeed equivalent.
		\end{proof}
	
		Five real numbers in $E_{\sqrt{2}}$ include 0, 1, 2, 3, and 4.
	
	\end{enumerate}


\end{document}